\documentclass[a4paper,12pt]{article}

\usepackage{graphicx}
\usepackage{caption}
\usepackage{subcaption}
\usepackage{tikz}
\usepackage{pgf}
\usepackage{amsmath}
\usetikzlibrary{arrows.meta}
\usepackage[utf8]{inputenc}
\usepackage[english,greek]{babel}
\usepackage{hyperref}

\title{Προσομοίωση και Μοντελοποίηση \newline Δυναμικών Συστημάτων \newline Εργασία 1}
\author{Ρουσομάνης Γεώργιος (ΑΕΜ: 10703)}
\date{Απρίλιος 2025}

\begin{document}

\maketitle

\section*{Εισαγωγή}
Σκοπός της εργασίας είναι η εκτίμηση αγνώστων παραμέτρων με χρήση της μεθόδου μέγιστης καθόδου και της 
μεθόδου σχεδίασης κατά \selectlanguage{english}Lyapunov\selectlanguage{greek} (παράλληλη και μικτή τοπολογία).
Οι μέθοδοι αυτοί ονομάζονται μέθοδοι πραγματικού χρόνου ή \selectlanguage{english}online\selectlanguage{greek}
γιατί δίνουν εκτίμηση μέσα σε χρόνο μιας δειγματοληψίας. Για την εκτίμηση αυτή, οι αλγόριθμοι πραγματικού 
χρόνου στηρίζονται στην τρέχουσα τιμή και στην αμέσως προηγούμενη, δουλεύουν δηλαδή αναδρομικά. Απευθύνονται
κυρίως σε εφαρμογές στις οποίες ο αλγόριθμος πρέπει να είναι υπολογιστικά απλός και η διαδικασία της εκτίμησης 
να ολοκληρωθεί πριν λάβει χώρα η επόμενη δειγματοληψία. Χαρακτηριστικά παραδείγματα τέτοιων εφαρμογών είναι ο 
προσαρμοστικός έλεγχος συστημάτων, η πρόβλεψη της απόκρισης συστημάτων, η ανίχνευση και αναγνώριση βλαβών σε 
δυναμικά συστήματα. Η εργασία επικεντρώνεται στην υλοποίηση των μεθόδων στο 
\selectlanguage{english}MATLAB\selectlanguage{greek}, την κατάλληλη επιλογή των παραμέτρων σύμφωνα με την 
λογική του \selectlanguage{english}trial and error\selectlanguage{greek} και την επίδραση του θορύβου στις
εκτιμήσεις.

\section*{Θέμα 1}
Το σύστημα που μελετούμε είναι το σύστημα μάζας-ελατηρίου-αποσβεστήρα με εξωτερική δύναμη, η εξίσωση του
οποίου δίνεται από τη σχέση:
\begin{equation}
    m \ddot{x}(t) + b \dot{x}(t) + k x(t) = u(t),
    \label{eq:system_ODE}
\end{equation}
όπου $x(t)$ \selectlanguage{english}[m]\selectlanguage{greek} η μετατόπιση, $m>0$ η μάζα, $b>0$ ένας σταθερός 
συντελεστής απόσβεσης, $k>0$ η σταθερά του ελατηρίου, και $u(t)$ η εξωτερική δύναμη. Θεωρούμε ότι οι πραγματικές
τιμές των παραμέτρων είναι $m=1.315$, $b=0.225$ και $k=0.725$. Θεωρούμε επίσης πως οι καταστάσεις $x(t)$, 
$\dot{x}(t)$ και η είσοδος $u(t)$ είναι μετρήσιμες. Θα εξετάσουμε τις περιπτώσεις όπου $u(t) = 2.5$ και 
$u(t) = 2.5\sin t$.

Πρώτο βήμα στην ανάλυσή μας είναι η προσομοίωση της απόκρισης του συστήματος με κάποια συνάρτηση 
\selectlanguage{english}ODE solver\selectlanguage{greek} του 
\selectlanguage{english}MATLAB\selectlanguage{greek}. Η παραπάνω ΔΕ περιγράφει ένα γραμμικό και χρονικά 
αμετάβλητο σύστημα (ΓΧΑ). Συνεπώς, μπορεί να αναπαρασταθεί από τις εξισώσεις κατάστασης:
\begin{equation}
\begin{aligned}
    \dot{x}(t) = A x(t) + B u(t) \\
    y(t) = C x(t) + D u(t)
\end{aligned}
\label{eq:state_space_equations}
\end{equation}
όπου εδώ $x(t)$ είναι το διάνυσμα καταστάσεων και $y$ η έξοδος του συστήματος.

Ξεκινώντας από την (\ref{eq:system_ODE}) έχουμε:
\begin{equation}
    \ddot{x}(t) = -\frac{b}{m} \dot{x}(t) -\frac{k}{m} x(t) + \frac{1}{m} u(t).
    \label{eq:system_ODE_2}
\end{equation}
Θέτοντας $x_1 = x(t)$ και $x_2 = \dot{x}(t)$, δηλαδή $\ddot{x}(t) = \dot{x}_2$, προκύπτει:
\begin{equation*}
\begin{aligned}
    &\dot{x}_1 = x_2 \\
    &\dot{x}_2 = -\frac{b}{m} x_2 -\frac{k}{m} x_1 + \frac{1}{m} u(t).
\end{aligned}
\end{equation*}
Θεωρώντας ως έξοδο $y$ την μετατόπιση $x(t)$, οι εξισώσεις κατάστασης γράφονται:
\begin{equation*}
\begin{aligned}
    \begin{bmatrix}
        \dot{x}_1 \\
        \dot{x}_2
    \end{bmatrix} &= 
    \begin{bmatrix}
        0 & 1 \\
        -\frac{b}{m} & -\frac{k}{m}
    \end{bmatrix} \cdot
    \begin{bmatrix}
        x_1 \\
        x_2
    \end{bmatrix} +
    \begin{bmatrix}
        0 \\
        \frac{1}{m}
    \end{bmatrix} u(t) \\
    y &= 
    \begin{bmatrix}
        1 & 0
    \end{bmatrix} \cdot
    \begin{bmatrix}
        x_1 \\
        x_2
    \end{bmatrix}
\end{aligned}
\end{equation*}
Άρα οι πίνακες $A, \, B, \, C, \, D$ είναι:
\begin{equation}
    A = 
    \begin{bmatrix}
        0 & 1 \\
        -\frac{k}{m} & -\frac{b}{m}
    \end{bmatrix}, \quad B = 
    \begin{bmatrix}
        0 \\
        \frac{1}{m}
    \end{bmatrix}, \quad C = 
    \begin{bmatrix}
        1 & 0
    \end{bmatrix}, \quad D = 0
    \label{eq:state_space_matrices}
\end{equation}
Για την επιλογή της κατάλληλης συνάρτησης \selectlanguage{english}ODE solver\selectlanguage{greek} είναι
σημαντικό να προσδιορίσουμε αν το σύστημά μας είναι άκαμτο ή μη-άκαμπτο. Η συνάρτηση μεταφοράς δίνεται από
τον τύπο:
\begin{equation*}
    H(s) = C(sI-A)^{-1}B + D
\end{equation*}
και με αντικατάσταση των πινάκων από την (\ref{eq:state_space_matrices}) παίρνουμε
\begin{equation}
    H(s) = \frac{1}{m} \frac{1}{s^2 + \frac{b}{m}s + \frac{k}{m}}
    \label{eq:transfer_function}
\end{equation}
Η γενική μορφή της συνάρτησης μεταφοράς δευτεροβάθμιου συστήματος δίνεται από:
\begin{equation}
    H(s) = \frac{\omega_n^2}{s^2 + 2 \zeta \omega_n + \omega_n^2}
    \label{eq:transfer_function_order_2}
\end{equation}
όπου $\omega_n$ η φυσική συχνότητα του συστήματος και $\zeta$ ο συντελεστής απόσβεσης.
Συγκρίνοντας τις σχέσεις (\ref{eq:transfer_function}) και (\ref{eq:transfer_function_order_2}) παίρνουμε:
\begin{equation*}
    \begin{aligned}
        &\omega_n = \sqrt{\frac{k}{m}} \\
        &\zeta = \frac{b}{2}\sqrt{\frac{1}{mk}}
    \end{aligned}
\end{equation*}
και με αντικατάσταση των $m=1.315$, $b=0.225$ και $k=0.725$ βρίσκουμε ότι $\omega_n = 0.7425$, 
$ \zeta = 0.1152$. Εφόσον ισχύει $\zeta < 1$ οι πόλοι του συστήματος είναι συζυγείς μιγαδικοί με σταθερά 
χρόνου $\tau = \frac{1}{\zeta \omega_n} = 11.7$ \selectlanguage{english}sec\selectlanguage{greek}. Επειδή
οι σταθερές χρόνου του συστήματος είναι της ίδιας τάξης μεγέθους, το σύστημα χαρακτηρίζεται ως μη-άκαμπτο.
Συνεπώς, επιλέγουμε την συνάρτηση \selectlanguage{english}ode45\selectlanguage{greek} με βήμα ολοκλήρωσης 
$dt = 0.01 < 2 \tau$.

\subsection*{Μέθοδος μέγιστης καθόδου}
Για την εκτίμηση των παραμέτρων $m$, $b$ και $k$ πρέπει αρχικά να φέρουμε το σύστημά μας σε γραμμικά
παραμετροποιήσιμη μορφή. Η (\ref{eq:system_ODE}) μπορεί να εκφραστεί ως:
\begin{equation}
    \ddot{x} = 
    \begin{bmatrix}
        \frac{b}{m} & \frac{k}{m} & \frac{1}{m}
    \end{bmatrix} \cdot
    \begin{bmatrix}
        -\dot{x} & -x & u
    \end{bmatrix}^T
    \label{eq:linearly_parameterized_form}
\end{equation}
Η παραπάνω εξίσωση περιέχει το $\ddot{x}$, το οποίο δεν είναι μετρήσιμο, γεγονός που καθιστά αδύνατη την άμεση 
εκτίμηση των παραμέτρων. Για να ξεπεράσουμε αυτό το πρόβλημα, θέτουμε $\ddot{x} = s^2x$ και εφαρμόζουμε 
και στα δύο μέλη το ευσταθές φίλτρο $\Lambda(s) = s^2 + \lambda_1 s + \lambda_2$. Τότε η εξίσωση 
(\ref{eq:linearly_parameterized_form}) μετασχηματίζεται ως εξής:
\begin{equation*}
    \begin{aligned}
        \frac{s^2}{\Lambda(s)}x &= 
        \begin{bmatrix}
            \frac{b}{m} & \frac{k}{m} & \frac{1}{m}
        \end{bmatrix} \cdot
        \begin{bmatrix}
            -\frac{\dot{x}}{\Lambda(s)} & -\frac{x}{\Lambda(s)} & \frac{u}{\Lambda(s)}
        \end{bmatrix}^T 
        \Rightarrow \\
        \left(1 - \frac{\lambda_1 s + \lambda_2}{\Lambda(s)}\right)x &= 
        \begin{bmatrix}
            \frac{b}{m} & \frac{k}{m} & \frac{1}{m}
        \end{bmatrix} \cdot
        \begin{bmatrix}
            -\frac{\dot{x}}{\Lambda(s)} & -\frac{x}{\Lambda(s)} & \frac{u}{\Lambda(s)}
        \end{bmatrix}^T 
        \Rightarrow \\
        x &= 
        \begin{bmatrix}
            \frac{b}{m} - \lambda_1 & \frac{k}{m} - \lambda_2 & \frac{1}{m}
        \end{bmatrix} \cdot
        \begin{bmatrix}
            -\frac{\dot{x}}{\Lambda(s)} & -\frac{x}{\Lambda(s)} & \frac{u}{\Lambda(s)}
        \end{bmatrix}^T 
    \end{aligned}
\end{equation*}
και περνώντας στο πεδίο του χρόνου:
\begin{equation}
    x(t) = \theta^{\star T}\phi(t)
    \label{eq:linearly_parameterized_form_2}
\end{equation}
όπου $\theta^{\star}$ είναι ένα άγνωστο σταθερό διάνυσμα που εμπεριέχει τις πραγματικές τιμές των παραμέτρων 
προς εκτίμηση και $\phi(t)$ το μετρήσιμο διάνυσμα οπισθοδρόμησης. Στόχος μας είναι η εύρεση μίας 
εκτίμησης $\hat{\theta}$ του $\theta^{\star}$ και έπειτα η εύρεση των εκτιμήσεων $\hat{m}$, $\hat{b}$, 
$\hat{k}$ για κάθε χρόνο $t$. Θεωρούμε το σύστημα αναγνώρισης:
\begin{equation}
    \hat{x}(t) = \hat{\theta}(t)\phi(t),
    \label{eq:identification_system}
\end{equation}
όπου η έξοδος $\hat{x}(t)$ αποτελεί εκτίμηση της εξόδου $x(t)$ του πραγματικού συστήματος 
(\ref{eq:linearly_parameterized_form_2}). Από τις (\ref{eq:linearly_parameterized_form_2}) και 
(\ref{eq:identification_system}) σχηματίζεται το σφάλμα αναγνώρισης:
\begin{equation}
    e = x - \hat{x} = x - \hat{\theta} \phi = (\theta^{\star} - \hat{\theta}) \phi
    \label{eq:identification_error}
\end{equation}
Η μέθοδος της μέγιστης καθόδου στηρίζεται για την εύρεση της αναδρομικής εκτίμησης $\hat{\theta}$ του 
$\theta^{\star}$ στην ελαχιστοποίηση ως προς $\hat{\theta}$ κατάλληλα ορισμένης συνάρτησης κόστους του $e$.
Η συνάρτηση κόστους που επιλέγουμε είναι:
\begin{equation}
    K(\hat{\theta}) = \frac{e^2}{2} = \frac{(x - \hat{\theta}\phi)^2}{2}.
    \label{eq:cost_function}
\end{equation}
Θέλουμε λοιπόν:
\begin{equation*}
    \arg \min_{\hat{\theta}} K(\hat{\theta}).
\end{equation*}
Η αναδρομική σχέση της νέας εκτίμησης $\hat{\theta}_{k+1}$ δίνεται από:
\begin{equation}
    \hat{\theta}_{k+1} = \hat{\theta}_k -\gamma \nabla K(\hat{\theta}) = 
    \hat{\theta}_k + \gamma(x - \hat{\theta}\phi)\phi = \hat{\theta}_k + \gamma e \phi,
    \quad \hat{\theta}(0) = \theta_0
    \label{eq:gradient_descend}
\end{equation}
όπου $\gamma > 0$ μία σταθερά, $\hat{\theta}(0)$ η αρχική τιμή του $\hat{\theta}$ και $\phi$, $e$ μετρήσιμα.
Έχοντας την εκτίμηση $\hat{\theta} = [\hat{\theta}_1\quad\hat{\theta}_2 \quad \hat{\theta}_3]^T$, τα
$\hat{m}, \, \hat{b}, \, \hat{k}$ προκύπτουν από τις σχέσεις:
\begin{equation*}
    \begin{aligned}
        \hat{m} = \frac{1}{\hat{\theta}_3}, \quad
        \hat{b} = \frac{\hat{\theta}_1 + \lambda_1}{\hat{\theta}_3}, \quad
        \hat{k} = \frac{\hat{\theta}_2 + \lambda_2}{\hat{\theta}_3}
    \end{aligned}
\end{equation*}

Θα μελετήσουμε την ευστάθεια του συστήματος (\ref{eq:identification_system}), (\ref{eq:gradient_descend}).
Για $m = 1.315$, $b = 0.225$, $k = 0.725$, το σύστημα (\ref{eq:system_ODE}) είναι ευσταθές, καθώς
οι πόλοι του βρίσκονται στο αριστερό ημιεπίπεδο. Επομένως, για φραγμένη είσοδο $u$, οι μεταβλητές κατάστασης
$x$ και $\dot{x}$ θα είναι επίσης φραγμένες, καθώς σχετίζονται γραμμικά με το $u$. Στην
(\ref{eq:linearly_parameterized_form_2}), το $\phi$ ορίζεται ως:
\begin{equation}
\phi = 
    \begin{bmatrix}
        -\frac{\dot{x}}{\Lambda(s)} & -\frac{x}{\Lambda(s)} & \frac{u}{\Lambda(s)}
    \end{bmatrix}^T. 
    \label{eq:regressor_vector}
\end{equation}
Εφόσον το $\Lambda(s)$ είναι ευσταθές φίλτρο και $x, \, \dot{x}, \, u \in L_{\infty}$ τότε θα ισχύει
$\phi \in L_{\infty}$. Παραγωγίζοντας την (\ref{eq:regressor_vector}) έχουμε:
\begin{equation}
\dot{\phi} = 
    \begin{bmatrix}
        -\frac{s}{\Lambda(s)}\dot{x} & -\frac{s}{\Lambda(s)}x & \frac{s}{\Lambda(s)}u
    \end{bmatrix}^T. 
    \label{eq:regressor_vector_derivative}
\end{equation}
Αφού το $\Lambda(s)$ είναι φίλτρο 2ης τάξης, ο βαθμός του αριθμητή της συνάρτησης μεταφοράς 
$\frac{s}{\Lambda(s)}$ θα είναι μικρότερος από τον βαθμό του παρανομαστή, άρα $\dot{\phi} \in L_{\infty}$.
Επομένως, σύμφωνα με το λήμμα \selectlanguage{english}Barbalat\selectlanguage{greek}, για εισόδους στην
(\ref{eq:linearly_parameterized_form_2}) που ικανοποιούν $\phi, \, \dot{\phi} \in L_{\infty}$, η 
(\ref{eq:gradient_descend}) εξασφαλίζει ότι η εκτιμώμενη έξοδος $\hat{x}(t)$ του συστήματος αναγνώρισης
$\hat{x}(t) = \hat{\theta}(t) u(t)$ θα συγκλίνει ασυμπτωτικά στην έξοδο $x(t)$ του πραγματικού συστήματος 
$x(t) = \theta^{\star}u(t)$, και ο ρυθμός μεταβολής των παραμέτρων $\dot{\hat{\theta}}(t)$ θα μειώνεται με τον
χρόνο και θα συγκλίνει ασυμπτωτικά στο μηδέν.

Στο Σχήμα~\ref{fig:task1_identification_error_gradient_descend} φαίνεται το σφάλμα αναγνώρισης $e(t)$, η 
πραγματική έξοδος του συστήματος $x(t)$ και η εκτίμησή της $\hat{x}(t)$, για $u(t) = 2.5$ με 
$\gamma = 10^{-5}$ (αριστερά) και για $u(t) = 2.5 \sin t$ με $\gamma = 10^{-3}$ (δεξιά). Παρατηρούμε ότι και 
στις δύο περιπτώσεις το σφάλμα μηδενίζεται μέσα σε εύλογο χρονικό διάστημα. Όπως έχει ήδη εξηγηθεί, η σύγκλιση
του $\hat{x}(t)$ στο $x(t)$ είναι εγγυημένη, καθώς και στις δύο περιπτώσεις ισχύει 
$u, \, \dot{u} \in L_{\infty}$, άρα $\phi, \, \dot{\phi} \in L_{\infty}$.

\begin{figure}[h!]
    \centering
    \begin{subfigure}{0.45\textwidth}
        \centering
        \includegraphics[width=\linewidth]{plot/task1_identification_error_gradient_descend_1.pdf}
        \caption{}
        \label{fig:task1_identification_error_gradient_descend_1}
    \end{subfigure}
    \hfill
    \begin{subfigure}{0.45\textwidth}
        \centering
        \includegraphics[width=\linewidth]{plot/task1_identification_error_gradient_descend_2.pdf}
        \caption{}
        \label{fig:task1_identification_error_gradient_descend_2}
    \end{subfigure}
    \caption{Σφάλμα αναγνώρισης για α) $u(t) = 2.5$ και β) $u(t) = 2.5 \sin t$}
    \label{fig:task1_identification_error_gradient_descend}
\end{figure}

Στο Σχήμα~\ref{fig:task1_parameter_estimations_gradient_descend} φαίνονται οι εκτιμήσεις των παραμέτρων για
$u(t) = 2.5$ με $\gamma = 10^{-5}$ (αριστερά) και για $u(t) = 2.5 \sin t$ με $\gamma = 10^{-3}$ (δεξιά). 
Παρατηρούμε ότι και στις δύο περιπτώσεις ο ρυθμός μεταβολής των παραμέτρων $\dot{\hat{\theta}}(t)$ μειώνεται με
τον χρόνο, όπως άλλωστε αναμέναμε από το λήμμα \selectlanguage{english}Barbalat\selectlanguage{greek}. Επίσης,
διαπιστώνουμε ότι για $u(t) = 2.5$ οι εκτιμήσεις των $m$ και $k$ απέχουν από τις πραγματικές τιμές τους κατά 
κάποια σταθερή ποσότητα. Αυτό μας προϊδεάζει ότι, ίσως σε αυτή την περίπτωση, το $\phi(t)$ δεν ικανοποιεί τη
ΣΕΔ. Αντιθέτως, για $u(t) = 2.5 \sin t$, οι εκτιμήσεις των παραμέτρων τείνουν ασυμπτωτικά να γίνουν ίσες με τις
πραγματικές τιμές, κάτι που αποτελεί ένδειξη ότι το $\phi(t)$ ικανοποιεί τη ΣΕΔ.

\begin{figure}[h!]
    \centering
    \begin{subfigure}{0.45\textwidth}
        \centering
        \includegraphics[width=\linewidth]{plot/task1_parameter_estimations_gradient_descend_1.pdf}
        \caption{}
        \label{fig:task1_parameter_estimations_gradient_descend_1}
    \end{subfigure}
    \hfill
    \begin{subfigure}{0.45\textwidth}
        \centering
        \includegraphics[width=\linewidth]{plot/task1_parameter_estimations_gradient_descend_2.pdf}
        \caption{}
        \label{fig:task1_parameter_estimations_gradient_descend_2}
    \end{subfigure}
    \caption{Εκτιμήσεις παραμέτρων για α) $u(t) = 2.5$ και β) $u(t) = 2.5 \sin t$}
    \label{fig:task1_parameter_estimations_gradient_descend}
\end{figure}

Θα εξετάσουμε αν το διάνυσμα $\phi(t)$ ικανοποιεί τη ΣΕΔ για τις δύο περιπτώσεις εισόδου. Λέμε ότι το διάνυσμα 
$\phi(t)$ ικανοποιεί τη Συνθήκη Εναπομείνουσας Διέγερσης (ΣΕΔ) με επίπεδο διέγερσης $\alpha_0 > 0$, αν
υπάρχουν σταθερές $\alpha_1, \, T_0 > 0$ τέτοιες ώστε:
\begin{equation}
    \alpha_1 I \ge \frac{1}{T_0} \int_{t}^{t+T_0} \phi(\tau) \phi^T(\tau) d\tau \ge \alpha_0 I, 
    \quad \forall t \ge 0
    \label{eq:PE}
\end{equation}
με $I$ τον μοναδιαίο πίνακα.

Όπως φαίνεται από την (\ref{eq:regressor_vector}), η μορφή του $\phi$ δεν επιτρέπει τον εκ των προτέρων
έλεγχο ικανοποίησης της (\ref{eq:PE}), γνωρίζοντας την είσοδο $u(t)$. Πρέπει δηλαδή εκ του αποτελέσματος να 
διαπιστώσουμε την ικανοποίηση της συνθήκης αυτής, άρα και τη σύγκλιση του $\hat{\theta}$ στο $\theta^{\star}$.

Στο Σχήμα~\ref{fig:task1_PE_eigenvalues_1} φαίνονται οι ιδιοτιμές του πίνακα 
$\int_{t}^{t+T_0} \phi(\tau) \phi^T(\tau) d\tau$ για $u(t) = 2.5$. Βλέπουμε ότι μία εκ των τριών ιδιοτιμών δεν
έχει άνω φράγμα, άρα η ΣΕΔ δεν ικανοποιείται, κάτι που συμφωνεί με τις παρατηρήσεις μας για το 
Σχήμα~\ref{fig:task1_parameter_estimations_gradient_descend_1}.

Στο Σχήμα~\ref{fig:task1_PE_eigenvalues_2} φαίνονται οι ιδιοτιμές του πίνακα 
$\int_{t}^{t+T_0} \phi(t) \phi(t)^T dt$ για $u(t) = 2.5 \sin t$. Σε αυτή την περίπτωση, βλέπουμε ότι όλες οι 
ιδιοτιμές είναι φραγμένες $\forall t \ge 0$, άρα ικανοποιείται η ΣΕΔ, κάτι που συμφωνεί με τα σχόλιά μας 
για το Σχήμα~\ref{fig:task1_parameter_estimations_gradient_descend_2}. 

Βλέπουμε επομένως ότι η ενέργεια του σήματος εισόδου $u(t) = 2.5 \sin t$ είναι τέτοια που να ενεργοποιεί με 
επειμένοντα τρόπο το σύστημα ώστε να μπορέσει να εμφανίσει την κρυμμένη πληροφορία $\theta^{\star}$.

\begin{figure}[h!]
    \centering
    \begin{subfigure}{0.45\textwidth}
        \centering
        \includegraphics[width=\linewidth]{plot/task1_PE_eigenvalues_1.pdf}
        \caption{}
        \label{fig:task1_PE_eigenvalues_1}
    \end{subfigure}
    \hfill
    \begin{subfigure}{0.45\textwidth}
        \centering
        \includegraphics[width=\linewidth]{plot/task1_PE_eigenvalues_2.pdf}
        \caption{}
        \label{fig:task1_PE_eigenvalues_2}
    \end{subfigure}
    \caption{Ιδιοτιμές πίνακα $\int_{t}^{t+T_0} \phi(\tau) \phi^T(\tau) d\tau$ για 
    α) $u(t) = 2.5$ και β) $u(t) = 2.5 \sin t$}
    \label{fig:task1_PE_eigenvalues}
\end{figure}

Στην (\ref{eq:gradient_descend}), η επιλογή του $\gamma$ έγινε με τη μέθοδο \selectlanguage{english}trial and 
error\selectlanguage{greek}. Γνωρίζουμε ότι όταν ισχύει η ΣΕΔ, αυξάνοντας την τιμή του $\gamma$, επιτυγχάνουμε 
ταχύτερη σύγκλιση του $\hat{\theta}(t)$ στο $\theta^{\star}$. Πολύ μεγάλες τιμές του $\gamma$, όμως, καθιστούν 
την (\ref{eq:gradient_descend}) άκαμπτη και επομένως δύσκολα αριθμητικά επιλύσιμη. Συνεπώς, οι τιμές του 
$\gamma$ επιλέχθηκαν ως οι μέγιστες δυνατές που δεν καθιστούν την (\ref{eq:gradient_descend}) άκαμπτη.

Στην ανάλυση που προηγήθηκε, φιλτράραμε το μη μετρίσιμο $\ddot{x}(t)$ με ένα ευσταθές φίλτρο 2ης τάξης. Αν 
χρησιμοποιούσαμε ένα ευσταθές φίλτρο 1ης τάξης, θα καταλήγαμε στη γραμμικά παραμετροποιήσιμη μορφή
$\dot{x}(t) = \theta^{\star T} \phi(t)$. Τότε το σύστημα αναγνώρισης θα ήταν 
$\hat{\dot{x}}(t) = \hat{\theta}(t)\phi(t)$ και το σφάλμα αναγνώρισης $e = \dot{x} - \hat{\dot{x}}$. 
Επομένως, σε αυτή την περίπτωση, η μέθοδος μέγιστης καθόδου θα μας έδινε την εκτίμηση $\hat{\theta}$ για την 
οποία η απόκλιση του $\hat{\dot{x}}$ από το $\dot{x}$ θα ήταν ελάχιστη.

%%%%%%%%%%%%%%%%%%%%%%%%%%%%%%%%%%%%%%%%%%%%%%%%%%%%%%%%%%%%%%%%%%%%%%%%%%%%%%%%%%%%%%%%%%%%%%%%%

\subsection*{Μέθοδος \selectlanguage{english}Lyapunov\selectlanguage{greek}}
Θα εκτιμήσουμε τις παραμέτρους $m$, $b$, $k$ με τη μέθοδο \selectlanguage{english}Lyapunov\selectlanguage{greek}
παράλληλης (Π) και μικτής (Μ) δομής, θεωρώντας ως είσοδο την $u(t) = 2.5 \sin t$.

Όπως ήδη έχουμε αναφέρει, το σύστημά μας μπορεί να γραφεί στην μορφή:
\begin{equation*}
    \dot{x} = Ax + Bu,
\end{equation*}
όπου οι πίνακες $A, \, B$ δίνονται από την (\ref{eq:state_space_matrices}) και $x$ το διάνυσμα καταστάσεων.
Προφανώς, ο πίνακας $A$ είναι αρνητικά ημιορισμένος καθώς οι ιδιοτιμές του βρίσκονται στο αρνητικό ημιεπίπεδο.

Θα μελετήσουμε πρώτα την Π δομή. Το σύστημα αναγνώρισης είναι:
\begin{equation}
    \dot{\hat{x}} = \hat{A}\hat{x} + \hat{B} u,
    \label{eq:lyapunov_parallel_identification_system}
\end{equation}
όπου $\hat{x}$ η εκτίμηση του διανύσματος καταστάσεων και $\hat{A}, \, \hat{B}$ οι εκτιμήσεις των $A, \, B$.
Ορίζουμε το σφάλμα αναγνώρισης $e = x - \hat{x}$. Παραγωγίζοντας το $e$ ως προς τον χρόνο προκύπτει η διαφορική
εξίσωση του σφάλματος:
\begin{equation}
    \dot{e} = Ax + Bu - \hat{A}\hat{x} - \hat{B}u.
    \label{eq:lyapunov_parallel_identification_error_derivative_1}
\end{equation}
Αν προσθαφαιρέσουμε τον όρο $A\hat{x}$ η (\ref{eq:lyapunov_parallel_identification_error_derivative_1}) γίνεται:
\begin{equation}
    \dot{e} = A(x - \hat{x}) - (\hat{A} - A)\hat{x} - (\hat{B} - B)u.
    \label{eq:lyapunov_parallel_identification_error_derivative_2}
\end{equation}
Ορίζουμε τα παραμετρικά σφάλματα $\tilde{A} = \hat{A} - A$, $\tilde{B} = \hat{B} - B$ και η 
(\ref{eq:lyapunov_parallel_identification_error_derivative_2}) παίρνει τη μορφή:
\begin{equation}
    \dot{e} = Ae -\tilde{A}\hat{x} - \tilde{B}u.
    \label{eq:lyapunov_parallel_identification_error_derivative_3}
\end{equation}
Επιλέγουμε την συνάρτηση \selectlanguage{english}Lyapunov\selectlanguage{greek}:
\begin{equation}
    V = \frac{1}{2}e^Te + \frac{1}{2}\mathrm{Tr}\{\tilde{A}^T\tilde{A}\} + 
    \frac{1}{2}\mathrm{Tr}\{\tilde{B}^T\tilde{B}\}
    \label{eq:lyapunov_function}
\end{equation}
όπου $\mathrm{Tr\{.\}}$ το ίχνος ενός πίνακα. Αν παραγωγίσουμε την (\ref{eq:lyapunov_function}) ως προς τον
χρόνο κατά μήκος της λύσης της (\ref{eq:lyapunov_parallel_identification_error_derivative_3}) βρίσκουμε έπειτα
από πράξεις:
\begin{equation}
    \dot{V} = e^TAe + \mathrm{Tr}\{\tilde{A}\dot{\hat{A}} + \tilde{B}\dot{\hat{B}} - 
    \tilde{A}\hat{x}e^T - \tilde{B}ue^T\}.
    \label{eq:lyapunov_parallel_function_derivative_1}
\end{equation}
Επιλέγοντας:
\begin{equation}
    \begin{aligned}
        \dot{\hat{A}} = \hat{x}e^T \\ 
        \dot{\hat{B}} = ue^T
    \end{aligned}
    \label{eq:lyapunov_parallel_update_formula}
\end{equation}
η (\ref{eq:lyapunov_parallel_function_derivative_1}) γίνεται:
\begin{equation}
    \dot{V} = e^TAe
    \label{eq:lyapunov_parallel_function_derivative_2}
\end{equation}
Παρατηρούμε ότι η (\ref{eq:lyapunov_parallel_update_formula}) είναι υλοποιήσιμη, καθώς τα $e$, $u$ είναι 
μετρήσιμα. Επιπλέον, για να είναι το σύστημα (\ref{eq:lyapunov_parallel_identification_system}), 
(\ref{eq:lyapunov_parallel_update_formula}) ευσταθές, πρέπει να ισχύει $\dot{V} = e^T A e \leq 0,\, \forall t$. 
Για να ισχύει αυτό για κάθε $e$, ο πίνακας $A$ θα πρέπει, εκτός από αρνητικά ημιορισμένος, να είναι και 
συμμετρικός - κάτι που στην περίπτωσή μας δεν ισχύει. Επομένως, δεν μπορούμε να εγγυηθούμε ότι 
$\lim_{t \to \infty} e(t) = 0$.

Στο Σχήμα~\ref{fig:task1_lyapunov_derivative_function_parallel} φαίνεται η παράγωγος της συνάρτησης 
\selectlanguage{english}Lyapunov\selectlanguage{greek}. Διαπιστώνουμε ότι δεν ισχύει 
$\dot{V} \leq 0, \, \forall t$, επαληθεύοντας τη θεωρητική μας ανάλυση.

Στο Σχήμα~\ref{fig:task1_identification_error_lyapunov_parallel} παρατηρούμε ότι το σφάλμα αναγνώρισης
δεν συγκλίνει στο μηδέν. Αυτό αιτιολογείται από το γεγονός ότι η παράγωγος της συνάρτησης 
\selectlanguage{english}Lyapunov\selectlanguage{greek} λαμβάνει και θετικές τιμές, καθιστώντας το σύστημα
αναγνώρισης ασταθές.

Στο Σχήμα~\ref{fig:task1_parameter_estimations_lyapunov_parallel} παρουσιάζονται οι εκτιμήσεις των παραμέτρων
για τη Π δομή. Παρατηρούμε ότι οι εκτιμήσεις αποκλίνουν όταν η παράγωγος της συνάρτησης 
\selectlanguage{english}Lyapunov\selectlanguage{greek} γίνεται θετική.

Για τη Μ δομή εργαζόμαστε αντίστοιχα. Το σύστημα αναγνώρισης είναι:
\begin{equation}
    \dot{\hat{x}} = \hat{A}x + \hat{B} u + C(x - \hat{x}),
    \label{eq:lyapunov_mixed_identification_system}
\end{equation}
όπου $C$ συμμετρικός και θετικά ημιορισμένος πίνακας. Η παράγωγος του σφάλματος αναγνώρισης $e = x - \hat{x}$ 
βρίσκεται:
\begin{equation}
    \dot{e} = -\tilde{A}x - \tilde{B}u - Ce
    \label{eq:lyapunov_mixed_identification_error_derivative}
\end{equation}
Η συνάρτηση \selectlanguage{english}Lyapunov\selectlanguage{greek} θα δίνεται και πάλι από την 
(\ref{eq:lyapunov_function}). Παραγωγίζοντας την (\ref{eq:lyapunov_function}) και αντικαθιστώντας
την (\ref{eq:lyapunov_mixed_identification_error_derivative}) προκύπτει:
\begin{equation}
    \dot{V} = -e^TCe + \mathrm{Tr}\{\tilde{A}\dot{\hat{A}} + \tilde{B}\dot{\hat{B}} - 
    \tilde{A}xe^T - \tilde{B}ue^T\}.
    \label{eq:lyapunov_mixed_function_derivative_1}  
\end{equation}
Επιλέγουμε τώρα:
\begin{equation}
    \begin{aligned}
        \dot{\hat{A}} = xe^T \\ 
        \dot{\hat{B}} = ue^T
    \end{aligned}
    \label{eq:lyapunov_mixed_update_formula}
\end{equation}
και η (\ref{eq:lyapunov_mixed_function_derivative_1}) γίνεται:
\begin{equation}
    \dot{V} = -e^TCe
    \label{eq:lyapunov_mixed_function_derivative_2}
\end{equation}
Βλέπουμε ότι στη μικτή δομή ισχύει $\dot{V} \leq 0, \, \forall t$, καθώς ο $C$ είναι συμμετρικός και θετικά
ημιορισμένος. Συνεπώς, το σύστημα (\ref{eq:lyapunov_mixed_identification_system}), 
(\ref{eq:lyapunov_mixed_update_formula}) είναι ευσταθές.

Η εκ των προτέρων διασφάλιση της ευστάθειας του συστήματος εκτίμησης στη Π δομή είναι πρακτικά αδύνατη,
καθώς ο πίνακας $A$ είναι άγνωστος. Αντίθετα, στη Μ δομή μπορούμε να εγγυηθούμε την ευστάθεια με κατάλληλη 
επιλογή του πίνακα $C$.

Στην ανάλυσή μας θέσαμε $C = \psi I$, όπου $I$ είναι ο μοναδιαίος πίνακας και $\psi > 0$ το κέρδος. 
Η επιλογή του $\psi$ έγινε και πάλι με την τεχνική 
\selectlanguage{english}trial and error\selectlanguage{greek}. Μεγαλύτερες τιμές του $\psi$ έχουν ως 
αποτέλεσμα να υπερισχύει ο διορθωτικός όρος $C(x - \hat{x})$ στην 
(\ref{eq:lyapunov_mixed_identification_system}), οδηγώντας σε ταχύτερη σύγκλιση. Ωστόσο, πολύ μεγάλες τιμές 
του $\psi$ ενδέχεται να καταστήσουν το σύστημα άκαμπτο. Τελικά, επιλέγουμε $\psi = 100$.

Στο Σχήμα~\ref{fig:task1_lyapunov_derivative_function_mixed} φαίνεται η παράγωγος της συνάρτησης 
\selectlanguage{english}Lyapunov\selectlanguage{greek}. Πράγματι, ισχύει 
$\dot{V} \leq 0, \, \forall t$, καθιστώντας το σύστημα αναγνώρισης ευσταθές.

Στο Σχήμα~\ref{fig:task1_identification_error_lyapunov_mixed} παρατηρούμε ότι το σφάλμα αναγνώρισης
συγκλίνει, και μάλιστα πολύ γρήγορα, στο μηδέν, λόγω της μεγάλης τιμής του κέρδους $\psi$ που επιλέξαμε.

Στο Σχήμα~\ref{fig:task1_parameter_estimations_lyapunov_mixed} παρουσιάζονται οι εκτιμήσεις των παραμέτρων
για τη Π δομή. Παρατηρούμε ότι οι εκτιμήσεις συγκλίνουν προς μία σταθερή τιμή, η οποία ωστόσο διαφέρει
από την πραγματική. Αυτό αιτιολογείται από το γεγονός ότι η είσοδος $u(t) = 2.5 \sin t$ δεν ικανοποιεί τη
ΣΕΔ.

\begin{figure}[h!]
    \centering
    \begin{subfigure}{0.45\textwidth}
        \centering
        \includegraphics[width=\linewidth]{plot/task1_lyapunov_derivative_function_parallel.pdf}
        \caption{}
        \label{fig:task1_lyapunov_derivative_function_parallel}
    \end{subfigure}
    \hfill
    \begin{subfigure}{0.45\textwidth}
        \centering
        \includegraphics[width=\linewidth]{plot/task1_lyapunov_derivative_function_mixed.pdf}
        \caption{}
        \label{fig:task1_lyapunov_derivative_function_mixed}
    \end{subfigure}
    \caption{Παράγωγος συνάρτησης \selectlanguage{english}Lyapunov\selectlanguage{greek} για 
    α) παράλληλη δομή β) μεικτή δομή}
    \label{fig:task1_lyapunov_derivative_function}
\end{figure}

\begin{figure}[h!]
    \centering
    \begin{subfigure}{0.45\textwidth}
        \centering
        \includegraphics[width=\linewidth]{plot/task1_identification_error_lyapunov_parallel.pdf}
        \caption{}
        \label{fig:task1_identification_error_lyapunov_parallel}
    \end{subfigure}
    \hfill
    \begin{subfigure}{0.45\textwidth}
        \centering
        \includegraphics[width=\linewidth]{plot/task1_identification_error_lyapunov_mixed.pdf}
        \caption{}
        \label{fig:task1_identification_error_lyapunov_mixed}
    \end{subfigure}
    \caption{Σφάλμα αναγνώρισης της μεθόδου \selectlanguage{english}Lyapunov\selectlanguage{greek} για 
    α) παράλληλη δομή β) μεικτή δομή}
    \label{fig:task1_identification_error_lyapunov}
\end{figure}

\begin{figure}[h!]
    \centering
    \begin{subfigure}{0.45\textwidth}
        \centering
        \includegraphics[width=\linewidth]{plot/task1_parameter_estimations_lyapunov_parallel.pdf}
        \caption{}
        \label{fig:task1_parameter_estimations_lyapunov_parallel}
    \end{subfigure}
    \hfill
    \begin{subfigure}{0.45\textwidth}
        \centering
        \includegraphics[width=\linewidth]{plot/task1_parameter_estimations_lyapunov_mixed.pdf}
        \caption{}
        \label{fig:task1_parameter_estimations_lyapunov_mixed}
    \end{subfigure}
    \caption{Εκτιμήσεις παραμέτρων της μεθόδου \selectlanguage{english}Lyapunov\selectlanguage{greek} για 
    α) παράλληλη δομή β) μεικτή δομή}
    \label{fig:task1_parameter_estimations_lyapunov}
\end{figure}


\subsection*{Μέθοδος \selectlanguage{english}Lyapunov\selectlanguage{greek} και θόρυβος δειγματοληψίας}

Θεωρούμε ότι η έξοδος $x(t)$ μετριέται με θόρυβο $\eta(t) = \eta_0 \sin(2\pi f_0 t), \, \forall t \geq 0$, 
όπου $\eta_0 = 0.25$ και $f_0 = 20$. Θα μελετήσουμε την επίδραση του θορύβου στις εκτιμήσεις των παραμέτρων
και της εξόδου για τη μέθοδο \selectlanguage{english}Lyapunov\selectlanguage{greek} με παράλληλη και μικτή
δομή.

Στο Σχήμα~\ref{fig:task1_identification_error_lyapunov_parallel_noise} παρουσιάζεται η απόλυτη τιμή του 
σφάλματος αναγνώρισης, με και χωρίς προσθήκη θορύβου στην έξοδο, για τη Π δομή. Παρατηρούμε ότι η προσθήκη
θορύβου έχει αμελητέα επίδραση στο σφάλμα αναγνώρισης. Αντίθετα, στο 
Σχήμα~\ref{fig:task1_identification_error_lyapunov_mixed_noise} παρατηρούμε ότι στη Μ δομή η προσθήκη θορύβου
επηρεάζει σημαντικά το σφάλμα αναγνώρισης. Αυτό μπορεί να αιτιολογηθεί ως εξής. Από τις
(\ref{eq:lyapunov_parallel_update_formula}), (\ref{eq:lyapunov_mixed_update_formula}) προκύπτει:

\begin{equation*}
    \begin{aligned}
        \dot{\hat{A}} = \hat{x}e^T = \hat{x}(x^T - \hat{x}^T) \quad \text{Π δομή} \\
        \dot{\hat{A}} = xe^T = x(x^T - \hat{x}^T) \quad \text{Μ δομή}
    \end{aligned}
\end{equation*}
Αν στη σχέση αυτή μία από τις μεταβλητές κατάστασης του πραγματικού συστήματος διαβρωθεί με προσθετικό
θόρυβο $\eta(t)$, τότε στη Μ δομή η εκτίμηση $\hat{A}$ θα εξαρτάται τόσο από το $\eta^2(t)$ όσο και από το 
$\eta(t)$. Αντίθετα, στη Π δομή η εξάρτηση εμφανίζεται μόνο ως προς το $\eta(t)$. Συνεπώς, η Μ δομή είναι
πιο ευαίσθητη στην παρουσία θορύβου σε σύγκριση με τη Π δομή.

Στο Σχήμα~\ref{fig:task1_estimation_parameter_error_vs_noise_amplitude_lyapunov} παρουσιάζεται το σφάλμα 
εκτίμησης των παραμέτρων συναρτήσει του πλάτους του θορύβου, για την παράλληλη (αριστερά) και τη μικτή (δεξιά) 
δομή. Παρατηρούμε ότι στην παράλληλη δομή το σφάλμα εκτίμησης παραμένει ουσιαστικά ανεξάρτητο από το πλάτος του
θορύβου, ενώ στη μικτή δομή το σφάλμα αυξάνεται με την αύξηση του πλάτους του θορύβου. 

Η μειωμένη ευαισθησία στο πλάτος του θορύβου στη Π δομή μπορεί να αιτιολογηθεί παρατηρώντας το σύστημα 
αναγνώρισης (\ref{eq:lyapunov_parallel_identification_system}), όπου διαπιστώνουμε ότι δεν 
χρησιμοποιούνται οι πραγματικές τιμές των καταστάσεων του συστήματος. Αντίθετα, στη Μ δομή, από τη 
σχέση~(\ref{eq:lyapunov_mixed_identification_system}), προκύπτει ότι χρησιμοποιούνται απευθείας οι μετρήσεις 
των καταστάσεων του πραγματικού συστήματος, γεγονός που εξηγεί την αυξημένη επίδραση του πλάτους του θορύβου.

\begin{figure}[h!]
    \centering
    \begin{subfigure}{0.45\textwidth}
        \centering
        \includegraphics[width=\linewidth]{plot/task1_identification_error_lyapunov_parallel_noise.pdf}
        \caption{}
        \label{fig:task1_identification_error_lyapunov_parallel_noise}
    \end{subfigure}
    \hfill
    \begin{subfigure}{0.45\textwidth}
        \centering
        \includegraphics[width=\linewidth]{plot/task1_identification_error_lyapunov_mixed_noise.pdf}
        \caption{}
        \label{fig:task1_identification_error_lyapunov_mixed_noise}
    \end{subfigure}
    \caption{Σφάλμα εκτίμησης με και χωρίς θόρυβο εξόδου για την μέθοδο
    \selectlanguage{english}Lyapunov\selectlanguage{greek} με 
    α) παράλληλη δομή β) μεικτή δομή}
    \label{fig:task1_identification_error_lyapunov_noise}
\end{figure}

\begin{figure}[h!]
    \centering
    \begin{subfigure}{0.45\textwidth}
        \centering
        \includegraphics[width=\linewidth]{
        plot/task1_estimation_parameter_error_vs_noise_amplitude_lyapunov_parallel.pdf}
        \caption{}
        \label{fig:task1_estimation_parameter_error_vs_noise_amplitude_lyapunov_parallel}
    \end{subfigure}
    \hfill
    \begin{subfigure}{0.45\textwidth}
        \centering
        \includegraphics[width=\linewidth]{
        plot/task1_estimation_parameter_error_vs_noise_amplitude_lyapunov_mixed.pdf}
        \caption{}
        \label{fig:task1_estimation_parameter_error_vs_noise_amplitude_lyapunov_mixed}
    \end{subfigure}
    \caption{Σφάλματα εκτίμησης παραμέτρων συναρτήσει του πλάτους του θορύβου για την μέθοδο
    \selectlanguage{english}Lyapunov\selectlanguage{greek} με 
    α) παράλληλη δομή β) μεικτή δομή}
    \label{fig:task1_estimation_parameter_error_vs_noise_amplitude_lyapunov}
\end{figure}

\section*{Θέμα 2}

Το μη-γραμμικό σύστημα που περιγράφει τη γωνία κύλισης 
\selectlanguage{english}(roll angle)\selectlanguage{greek} ενός αεροσκάφους δίνεται από τη διαφορική εξίσωση:
\begin{equation*}
    \ddot{r}(t) = -a_1 \dot{r}(t) - a_2 \sin(r(t)) + a_3 \dot{r}^2(t) \sin(2r(t)) + b u(t) + d(t)
\end{equation*}
όπου $r(t)$ η γωνία κύλισης \selectlanguage{english}[rad]\selectlanguage{greek},
$a_i, \, i = 1,\,2,\,3$ και $b$ σταθερές, άγνωστες παράμετροι, $u(t)$ η είσοδος ελέγχου και $d(t)$
οι εξωτερικές διαταραχές.

\subsection*{Σχεδίαση ελεγκτή ανάδρασης}

Αρχικά θα σχεδιάσουμε ελεγκτή ανάδρασης ώστε η γωνία $r(t)$ να ακολουθήσει την επιθυμητή τροχιά $r_d(t)$, η
οποία ορίζεται ώστε:
\begin{equation}
    r_d(t) = -\frac{\pi}{1000}t^2 + \frac{\pi}{50}t, \quad t \in [0,\,20].
    \label{eq:desired_trajectory}
\end{equation}
Από την (\ref{eq:desired_trajectory}) μπορούμε να επαληθεύσουμε ότι 
$r_d(0) = 0, \, r_d(10) = \pi / 10, \, r_d(20) = 0$.
Ο ελεγκτής ανάδρασης που θα χρησιμοποιήσουμε είναι:
\begin{equation*}
    \begin{aligned}
    z_1(t) &= \frac{r(t) - r_d(t)}{\phi(t)} \\
    \alpha(t) &= -k_1 T(z_1(t)) \\
    z_2(t) &= \frac{\dot{r}(t) - \alpha(t)}{\rho} \\
    u(t) &= -k_2 T(z_2(t))
    \end{aligned}
\end{equation*}
όπου $\phi(t) = (\phi_0 - \phi_\infty) e^{-\lambda t} + \phi_\infty$ η επιθυμητή ακρίβεια παρακολούθησης
με παραμέτρους $\phi_0 > \phi_{\infty} > 0$, $\lambda > 0$, $\phi_0 \gg |r(0) - r_d(0)|$ και
$T(z) = \ln\left( \frac{1+z}{1-z} \right)$. Επίσης $\rho \gg |\dot{r}(0) - \alpha(0)|$ και $k_1 > 0$, $k_2 > 0$
κέρδη ελεύθερης επιλογής.

Στο Σχήμα~\ref{fig:task2_roll_angle_response} φαίνεται το γράφημα γωνίας κύλισης $r(t)$ και τροχιάς αναφοράς
$r_d(t)$ για $\phi_0 = 0.4, \, \phi_{\infty} = 0.05, \, \lambda = 0.4, \, \rho = 0.5, \, k_1 = 1.2, \,
k_2 = 2$.

Από το Σχήμα~\ref{fig:task2_position_tracking_accuracy} βλέπουμε ότι για την ακρίβεια παρακολούθησης
της επιθυμητής τροχίας ισχύει $|r(t) - r_d(t)| < \phi(t), \, \forall t \geq 0$. Ομοίως από το 
Σχήμα~\ref{fig:task2_velocity_tracking_accuracy} για την ακρίβεια παρακολούθησης της επιθυμητής ταχύτητας 
ισχύει $|\dot{r}(t) - \alpha(t)| < \rho, \, \forall t \geq 0$. Συνεπώς η υλοποίησή μας είναι ορθή για αυτές
τις επιλογές των παραμέτρων.

Από το Σχήμα~\ref{fig:task2_position_tracking_accuracy_vs_phi_infty} παρατηρούμε ότι με τη μείωση του 
$\phi_{\infty}$ βελτιώνεται η ακρίβεια παρακολούθησης της επιθυμητής τροχίας $r_d(t)$.

\begin{minipage}[t]{0.45\textwidth}
    \centering
    \includegraphics[width=\linewidth]{plot/task2_roll_angle_response.pdf}
    \captionof{figure}{Γράφημα γωνίας κύλισης $r(t)$ και τροχιάς αναφοράς $r_d(t)$
    ($\phi_0 = 0.4, \, \phi_{\infty} = 0.05, \, \lambda = 0.4, \, \rho = 0.5, \, k_1 = 1.2, \, k_2 = 2$)}
    \label{fig:task2_roll_angle_response}
\end{minipage}%
\hfill
\begin{minipage}[t]{0.45\textwidth}
    \centering
    \includegraphics[width=\linewidth]{plot/task2_position_tracking_accuracy_vs_phi_infty.pdf}
    \captionof{figure}{Απόκλιση $|r(t) - r_d(t)|$ για διάφορες τιμές του 
    $\phi_{\infty}$}
    \label{fig:task2_position_tracking_accuracy_vs_phi_infty}
\end{minipage}

\begin{minipage}[t]{0.45\textwidth}
    \centering
    \includegraphics[width=\linewidth]{plot/task2_position_tracking_accuracy.pdf}
    \captionof{figure}{Απόκλιση $|r(t) - r_d(t)|$ 
    ($\phi_0 = 0.4, \, \phi_{\infty} = 0.05, \, \lambda = 0.4, \, \rho = 0.5, \, k_1 = 1.2, \, k_2 = 2$)}
    \label{fig:task2_position_tracking_accuracy}
\end{minipage}%
\hfill
\begin{minipage}[t]{0.45\textwidth}
    \centering
    \includegraphics[width=\linewidth]{plot/task2_velocity_tracking_accuracy.pdf}
    \captionof{figure}{Απόκλιση $|\dot{r}(t) - \alpha(t)|$ 
    ($\phi_0 = 0.4, \, \phi_{\infty} = 0.05, \, \lambda = 0.4, \, \rho = 0.5, \, k_1 = 1.2, \, k_2 = 2$)}
    \label{fig:task2_velocity_tracking_accuracy}
\end{minipage}


\end{document}
